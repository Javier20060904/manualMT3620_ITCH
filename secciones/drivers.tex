\section{Drivers}
\subsection{Alto Nivel}
El añadir archivos de cabecera y configuraciones en el app manifest es similar entre todos los periféricos.
\subsubsection{ADC}
\begin{lstlisting}[language = C, firstnumber=0]
#include <applibs/adc.h>
\end{lstlisting}
\begin{lstlisting}[language = json, firstnumber=0]	
"Capabilities": 
{  
	"Adc": [ "$MACRO_DEL_ADC" ] 
}
\end{lstlisting}

\subsubsection{GPIO}
\begin{lstlisting}[language = C, firstnumber=0]
#include <applibs/gpio.h>
\end{lstlisting}
\begin{lstlisting}[language = json, firstnumber=0]	
"Capabilities": 
{  
	"Gpio": [ "$MACRO_DEL_GPIO" ] 
}
\end{lstlisting}

\subsubsection{I2C}
\begin{lstlisting}[language = C, firstnumber=0]
#include "applibs_versions.h"
#include <applibs/i2c.h>
\end{lstlisting}
\begin{lstlisting}[language = json, firstnumber=0]		"Capabilities":
{  
	"I2C": [ "$MACRO_DEL_I2C" ] 
}
\end{lstlisting}

\subsubsection{PWM}
\begin{lstlisting}[language = C, firstnumber=0]
#include <applibs/pwm.h>
\end{lstlisting}
\begin{lstlisting}[language = json, firstnumber=0]	
"Capabilities": 
{  
	"Pwm": [ "$MACRO_DEL_PWM" ] 
}
\end{lstlisting}

\subsubsection{UART}
\begin{lstlisting}[language = C, firstnumber=0]
#include "applibs_versions.h"
#include <applibs/uart.h>
\end{lstlisting}
\begin{lstlisting}[language = json, firstnumber=0]	
	"Capabilities": 
	{  
		"Uart": [ "$MACRO_DEL_UART" ] 
	}
\end{lstlisting}

\subsubsection{SPI}
\begin{lstlisting}[language = C, firstnumber=0]
#include "applibs_versions.h"
#include <applibs/spi.h>
\end{lstlisting}
\begin{lstlisting}[language = json, firstnumber=0]	
"Capabilities": {  
	"SpiMaster": [ "$MACRO_DEL_SPI_ISU0", $MACRO_DEL_SPI_ISU1" ]
}
\end{lstlisting}

\subsection{Tiempo Real}
Al desarrollar aplicaciones en tiempo real, se utiliza uno de los ARM Cortex M4 que tiene el integrado. Ya existen unos \hyperlink{https://github.com/CodethinkLabs/mt3620-m4-drivers/tree/master}{drivers} para la MT3620, o puedes usar los registros de memoria que están documentados (\hyperlink{https://d86o2zu8ugzlg.cloudfront.net/mediatek-craft/documents/mt3620/MT3620-Datasheet-v1.7.pdf}{datasheet} y
\hyperlink{https://d86o2zu8ugzlg.cloudfront.net/mediatek-craft/documents/MT3620-M4-User-Manual.pdf}{manual}) por el fabricante.

A la hora de elegir un periférico en el app manifest, es casi lo mismo. Ahora, para seleccionar un puerto, se elige el identificador ``maping'' del archivo mt3620.json (véase capítulo 6).

\subsubsection{ADC}
\begin{lstlisting}[language = json, firstnumber=0]	
"Capabilities": {
	"Adc": [ "ADC-CONTROLLER-0" ]  }
\end{lstlisting}

\subsubsection{GPIO}
\begin{lstlisting}[language = json, firstnumber=0]	
"Capabilities": {
	"Gpio": [ 8, 12 ]
}
\end{lstlisting}

\subsubsection{I2C}
\begin{lstlisting}[language = json, firstnumber=0]	
	"Capabilities": {
		"I2C": [ "ISU0" ]
	}
\end{lstlisting}

\subsubsection{I2S}
\begin{lstlisting}[language = json, firstnumber=0]	
	"Capabilities": {
		"I2sSubordinate": [ "ISU0" ]
	}
\end{lstlisting}

\subsubsection{PWM}
\begin{lstlisting}[language = json, firstnumber=0]	
	"Capabilities": 
	{  
		"Pwm": [ "PWM-CONTROLLER-0" ] 
	}
\end{lstlisting}

\subsubsection{SPI}
\begin{lstlisting}[language = json, firstnumber=0]	
	"Capabilities": {
		"SpiMaster": [ "ISU0", "ISU1" ] }
\end{lstlisting}

\subsubsection{UART}
\begin{lstlisting}[language = json, firstnumber=0]	
"Capabilities": {
	"Uart": [ "ISU0" ]
}
\end{lstlisting}


