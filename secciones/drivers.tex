\section{Periféricos}
En esta seccion se mostrara algunas especificaciones del periferico y el como anadirlos en los archivos de cabecera y configuraciones en el app manifest.\par
Al desarrollar aplicaciones en RT, se utiliza uno de los ARM Cortex M4 que tiene el integrado. Ya existen unos \href{https://github.com/CodethinkLabs/mt3620-m4-drivers/tree/master}{drivers} para la MT3620, o puedes usar los registros de memoria que están documentados (\href{https://d86o2zu8ugzlg.cloudfront.net/mediatek-craft/documents/mt3620/MT3620-Datasheet-v1.7.pdf}{datasheet} y
\href{https://d86o2zu8ugzlg.cloudfront.net/mediatek-craft/documents/MT3620-M4-User-Manual.pdf}{manual}) por el fabricante. A la hora de elegir un periférico en el app manifest, es casi lo mismo. Ahora, para seleccionar un puerto, se elige el identificador ``maping'' del archivo mt3620.json (véase capítulo 6).


\subsection{GPIO}
La plataforma contiene 76 pines programables, algunos están multiplexados con otras funciones.
\subsubsection{HL}

\begin{lstlisting}[language = C, firstnumber=0]
	#include <applibs/gpio.h>
\end{lstlisting}
\begin{lstlisting}[language = json, firstnumber=0]	
	"Capabilities": 
	{  
		"Gpio": [ "$MACRO_DEL_GPIO" ] 
	}
\end{lstlisting}
\subsubsection{RT}
\begin{lstlisting}[language = json, firstnumber=0]	
	"Capabilities": {
		"Gpio": [ 8, 12 ]
	}
\end{lstlisting}

\subsection{ADC}
La plataforma contiene 8 canales de 12 bits, usando un voltaje de referencia a 2.5 V internos o 1.8 V externos.
\subsubsection{HL}
\begin{lstlisting}[language = C, firstnumber=0]
	#include <applibs/adc.h>
\end{lstlisting}
\begin{lstlisting}[language = json, firstnumber=0]	
	"Capabilities": 
	{  
		"Adc": [ "$MACRO_DEL_ADC" ] 
	}
\end{lstlisting}
\subsubsection{RT}
\begin{lstlisting}[language = json, firstnumber=0]	
	"Capabilities": {
		"Adc": [ "ADC-CONTROLLER-0" ]  }
\end{lstlisting}

\subsection{UART}
La plataforma contiene 5 interfaces seriales que pueden ser configuradas como UART y 2 UART dedicadas, a una velocidad de hasta 3Mbps.
\subsubsection{HL}
\begin{lstlisting}[language = C, firstnumber=0]
	#include "applibs_versions.h"
	#include <applibs/uart.h>
\end{lstlisting}
\begin{lstlisting}[language = json, firstnumber=0]	
	"Capabilities": 
	{  
		"Uart": [ "$MACRO_DEL_UART" ] 
	}
\end{lstlisting}
\subsubsection{RT}
\begin{lstlisting}[language = json, firstnumber=0]	
	"Capabilities": {
		"Uart": [ "ISU0" ]
	}
\end{lstlisting}

\subsection{PWM}
La plataforma contiene 12 Salidas PWM.
\subsubsection{HL}
\begin{lstlisting}[language = C, firstnumber=0]
	#include <applibs/pwm.h>
\end{lstlisting}
\begin{lstlisting}[language = json, firstnumber=0]	
	"Capabilities": 
	{  
		"Pwm": [ "$MACRO_DEL_PWM" ] 
	}
\end{lstlisting}
\subsubsection{RT}
\begin{lstlisting}[language = json, firstnumber=0]	
	"Capabilities": 
	{  
		"Pwm": [ "PWM-CONTROLLER-0" ] 
	}
\end{lstlisting}

\subsection{SPI}
La plataforma contiene 5 interfaces seriales que pueden ser configuradas como SPI maestro o esclavo, a una velocidad de hasta 40MHz.
\subsubsection{HL}
\begin{lstlisting}[language = C, firstnumber=0]
	#include "applibs_versions.h"
	#include <applibs/spi.h>
\end{lstlisting}
\begin{lstlisting}[language = json, firstnumber=0]	
	"Capabilities": {  
		"SpiMaster": [ "$MACRO_DEL_SPI_ISU0", $MACRO_DEL_SPI_ISU1" ]
	}
\end{lstlisting}
\subsubsection{RT}
\begin{lstlisting}[language = json, firstnumber=0]	
	"Capabilities": {
		"SpiMaster": [ "ISU0", "ISU1" ] }
\end{lstlisting}

\subsection{I2C}
La plataforma contiene 5 interfaces seriales que pueden ser configuradas como I2C maestro o esclavo, a una velocidad de hasta 1MHz.
\subsubsection{HL}
\begin{lstlisting}[language = C, firstnumber=0]
	#include "applibs_versions.h"
	#include <applibs/i2c.h>
\end{lstlisting}
\begin{lstlisting}[language = json, firstnumber=0]		"Capabilities":
	{  
		"I2C": [ "$MACRO_DEL_I2C" ] 
	}
\end{lstlisting}
\subsubsection{RT}
\begin{lstlisting}[language = json, firstnumber=0]	
	"Capabilities": {
		"I2C": [ "ISU0" ]
	}
\end{lstlisting}
