\section{Herramientas a instalar}

\begin{itemize}
	\item 
	Puerto USB
	\item
	Sistema Operativo: Windows 10/11 o 
	distribucion \href{https://ubuntu.com/}{Ubuntu} 18.04, 20.04, 22.04
	\item 
	Lenguaje de programación: C
	\item 
	Entorno de Desarrollo: \href{https://visualstudio.microsoft.com/es/}{Visual Studio}, 
	\href{https://code.visualstudio.com/}{Visual Studio Code} o Terminal
	\item
	Azure Sphere SDK en \href{https://learn.microsoft.com/en-us/azure-sphere/install/install-sdk}{Windows} o \href{https://learn.microsoft.com/en-us/azure-sphere/install/install-sdk-linux}{Linux}
\end{itemize}

\subsection{Instalación}
\subsubsection{Windows}
\begin{enumerate}
	\item 
	Conectar el microcontrolador, Windows automaticamente instalara los drivers necesarios (En caso de que no se puede usar los drivers de \href{https://www.ftdichip.com/Drivers/VCP.htm}{FTDI})
	\item 
	Instalar el \href{https://aka.ms/AzureSphereSDKDownload/Windows}{SDK}
	\item 
	Instalar \href{https://visualstudio.microsoft.com/downloads/}{Visual Studio 2022}
	\item 
	Instalar la extensión de \href{https://marketplace.visualstudio.com/items?itemName=AzureSphereTeam.AzureSphereSDKforVisualStudio2022}{Azure Sphere} para Visual Studio
\end{enumerate}
\subsubsection{Linux}
\begin{enumerate}
	\item 
	\href{https://aka.ms/AzureSphereSDKInstall/Linux}{Descargar el Script de instlacion}
	\item 
	Descomprimir el archivo
	\item 
	
	Abrir la terminal en la locación del archivo .sh y ejecutar el siguiente comando:
	\begin{lstlisting}[language=bash]
$ sudo ./install_azure_sphere_sdk.sh
	\end{lstlisting}
	\item Seguir con el proceso de instalación
	
\end{enumerate}